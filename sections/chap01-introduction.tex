%!TEX root = ../thesis.tex

%%%%%%%%%%%%%%%%%%%%%%
%    Introduction    %
%%%%%%%%%%%%%%%%%%%%%%
\message{^^J ^^J INTRODUCTION ^^J ^^J}
\newchap{Introduction}\label{sec:introduction}

% STANDARD MODEL
Over the last 100 years, physicists have made large strides in understanding the natural laws that govern the universe as we know it. The smallest building blocks of matter and the way they interact at small scales have been encoded in the Standard Model (SM) of particle physics.
In 2012, the discovery of a major missing piece, the Higgs boson was announced by the ATLAS and CMS experiments~\cite{Higgs_discovery_2012_CMS,Higgs_discovery_2012_ATLAS,Higgs_discovery_2013_CMS} at the CERN LHC, which came more than 47 years after its prediction~\cite{Higgs_theory1,Higgs_theory2}.
Despite its many remarkable successes, there are both theoretical and experimental reasons that imply the SM is not the final and most fundamental theory of nature.

See \ttbar, \bbbar

% CONTENT
It has the following structure:
Chapter~\ref{sec:SM} discusses the SM and its flavor symmetries, after which Chapter~\ref{sec:BSM} motivates the search for hypothetical particles like {\LQs}, which are beyond the SM.
Chapter~\ref{sec:dectector} describes the LHC accelerator, CMS detector and the collection of proton-proton collision data.
Chapter~\ref{sec:objects} describes the way physics objects that are relevant for this thesis are reconstructed in CMS,
while Chapter~\ref{sec:tauh} focuses mainly on the reconstruction and identification of hadronically decayed $\PGt$ leptons.
After these, Chapter~\ref{sec:selections} discusses the selections.
Chapter~\ref{sec:results} presents the preliminary results.
A summary is given in Chapter~\ref{sec:conclusions}.
