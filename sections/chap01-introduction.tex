%!TEX root = ../thesis.tex

%%%%%%%%%%%%%%%%%%%%%%
%    Introduction    %
%%%%%%%%%%%%%%%%%%%%%%

\message{^^J ^^J INTRODUCTION ^^J ^^J} % print to log
\newchap{Introduction}\label{sec:introduction}

% STANDARD MODEL
Over the last 100 years, physicists have made large strides in understanding the natural laws that govern the universe as we know it. The Standard Model (SM) of particle physics describes matter and its interactions.

In 2012, the discovery of the Higgs boson was announced by the ATLAS and CMS experiments~\cite{Higgs_discovery_2012_CMS,Higgs_discovery_2012_ATLAS,Higgs_discovery_2013_CMS} at CERN's LHC.
The Higgs was 47 years before that~\cite{Higgs_theory1,Higgs_theory2}.
Despite its many successes, there are both theoretical and experimental reasons that imply the SM is not the final theory of Nature.

% CONTENT
This dissertation has the following structure:
Chapter~\ref{sec:SM} discusses the SM, after which Chapter~\ref{sec:BSM} motivates the search for new hypothetical particles.
Chapter~\ref{sec:dectector} describes the LHC accelerator, CMS detector and the data collection.
Chapter~\ref{sec:objects} describes the physics objects that are relevant for this thesis.
%After these, Chapter~\ref{sec:selections} discusses the selections.
%Chapter~\ref{sec:results} presents the results.
%A summary is given in Chapter~\ref{sec:conclusions}.
