% Author: Izaak Neutelings (February 2023)
% Inspired by:
%   https://en.wikipedia.org/wiki/File:Standard_Model_of_Elementary_Particles.svg
\documentclass[border=3pt,tikz]{standalone}
\usepackage{amsmath} % for \text
\usepackage{xfrac} % for \sfrac
\usepackage{bm} % for \bm
\usetikzlibrary{calc}
\usetikzlibrary{positioning}
\tikzset{>=latex} % for LaTeX arrow head

% FONT
\usepackage{sansmath} % for \sansmath
\renewcommand{\familydefault}{\sfdefault} % set sans serif font globally
\sansmath % set sans serif font globally

% UNSLANT GREEK LETTERS for particle symbols
% https://tex.stackexchange.com/questions/145926/upright-greek-font-fitting-to-computer-modern
% https://tex.stackexchange.com/questions/236915/adjust-custom-made-upright-greek-letters-when-used-in-subscripts
\usepackage{scalerel}
\newsavebox{\foobox}
\newcommand{\slantbox}[2][0]{\mbox{%
  \sbox{\foobox}{#2}%
  \hskip\wd\foobox
  \pdfsave
  \pdfsetmatrix{1 0 #1 1}%
  \llap{\usebox{\foobox}}%
  \pdfrestore
}}
\newcommand\unslant[2][-.25]{%
  %\mkern1.2mu%
  \ThisStyle{\slantbox[#1]{$\SavedStyle#2$}}%
  \mkern-2.2mu%
}
\newcommand\PGn[1]{\unslant\nu_{#1}\mkern-1.5mu} % neutrino

% UNITS
%\newcommand\MeV{\,\text{GeV}\mkern-1mu/\mkern-1muc^2} % HEP units
%\newcommand\MeV{\,\text{MeV}\mkern-1mu/\mkern-1muc^2} % HEP units
%\newcommand\eV{\,\text{eV}\mkern-1mu/\mkern-1muc^2} % HEP units
\newcommand\GeV{\,\text{GeV}} % natural units
\newcommand\MeV{\,\text{MeV}} % natural units
\newcommand\eV{\,\text{eV}} % natural units

% COLORS
\colorlet{mylightblue}{blue!60!cyan!80!black!15}
\colorlet{mypurple}{blue!50!red!70}
\colorlet{gaugecol}{red!90!black!70} % Wiki red
\colorlet{leptoncol}{green!80!black!70} % Wiki green
%\colorlet{quarkcol}{blue!70!red!50} % Wiki purple
\colorlet{quarkcol}{blue!85!cyan!95!black!55} % Wiki purple
\colorlet{scalarcol}{yellow!75!orange!93!black}
%\colorlet{tensorcol}{blue!60!cyan!80!black!15} % Wiki light blue
\colorlet{tensorcol}{blue!50!red!70} % Wiki light blue
\colorlet{groupcol}{orange!15}

% STYLES
\tikzstyle{header}=[black,midway,font=\bf,align=center,scale=0.6]
\tikzstyle{proplabel}=[black!70,scale=0.5] % label of properties
\tikzstyle{bflabel}=[font=\bf,inner sep=0.5pt,rotate=90]

% LAYERS
\pgfdeclarelayer{back} % to draw on background
\pgfsetlayers{back,main} % set order

% PARTICLE macro
\def\pw{0.94} % width/height of particle box
\tikzset{x={(1.5,0)},y={(0,1.6)}} % scale x, y axes differently
\tikzset{
  intgroup/.style={draw=#1!90!black!80,line width=0.5, % interaction groups
                   fill=#1,fill opacity=0.5},
  intgroup/.default=groupcol,
  pics/particle/.style n args={6}{ % particle boxes
    code={
      \tikzset{/tikz/pic opacity/.get=\OP}
      \begin{scope}[opacity=\OP]
      \coordinate (-sw) at (-\pw/2,-\pw/2);
      \coordinate (-nw) at (-\pw/2, \pw/2);
      \coordinate (-se) at ( \pw/2,-\pw/2);
      \coordinate (-ne) at ( \pw/2, \pw/2);
      \ifnum\pgfkeysvalueof{/tikz/fillbox}=1
        \draw[#1,line width=1.1,rounded corners=3pt,shading angle=30,
              top color=#1!90!black!40,bottom color=#1!75!black!40]
          (-sw) rectangle (-ne);
      \else
        %\draw[draw=#1,line width=1.1,rounded corners=3pt]
        %  (-sw) rectangle (-ne);
        \fill[top color=#1,bottom color=#1!90!black,shading angle=30,
              rounded corners=3pt,even odd rule]
          (-0.48*\pw,-0.48*\pw) rectangle (0.48*\pw,0.48*\pw)
          [rounded corners=3.7pt] (-sw) rectangle (-ne);
      \fi
      \node[circle,ball color=#1,text=black,
            minimum size=20,inner sep=0,scale=1,
            postaction={fill=#1!77,opacity=0.8*\OP,
            draw=#1!80!black!90,ultra thin}]
        (-p) at (0.02,0.06) {\textbf{\boldmath{#2}}};
      \node[align=center,font=\bf,
            scale=\pgfkeysvalueof{/tikz/textscale}]
        at (0,-0.3) {#3};
      \node[below right,proplabel]
        at (-0.48*\pw,0.50*\pw) {\strut$#4$};
      \node[below right,proplabel]
        at (-0.48*\pw,0.35*\pw) {\strut$#5$};
      \node[below right,proplabel]
        at (-0.48*\pw,0.20*\pw) {\strut$#6$};
      \end{scope}
    }
  },
  textscale/.initial=0.8, % text scale
  fillbox/.initial=0, % fill particle boxes
  pic opacity/.initial=1 % opacity of pictures
}
\newcommand\opGen[2]{min(#1,(\setGen==0 || \setGen==#2) ? 1 : \d)} % calculate fermion opacity
%\def\opGen#1#2{ min(#1,(\setGen==0 || \setGen==#2) ? 1 : \d) } % calculate fermion opacity

\begin{document}


% SM PARTICLES: GROUPS (like Wiki)
% https://en.wikipedia.org/wiki/File:Standard_Model_of_Elementary_Particles.svg
%\def\f{1}
%\def\opQua{1} \def\opLep{1} \def\opNu{1}
%\def\opGlu{1} \def\opGam{1} \def\opWeak{1} \def\opHig{1}
%\def\setGen{3}
\def\d{0.1} % dimmed opacity to highlight others
\foreach \f in {0,1}{ % fill particle boxes
\foreach \opQua/\opLep/\opNu/\opGlu/\opGam/\opWeak/\opHig/\setGen in {%
  % highlight different groups of particles,
  % by reducing the opacity of others
  1/1/1/1/1/1/1/0,       % highlight everything
  1/1/1/\d/\d/\d/\d/0,   % fermions
  1/\d/\d/\d/\d/\d/\d/0, % quarks
  \d/1/1/\d/\d/\d/\d/0,  % leptons
  1/1/1/\d/\d/\d/\d/1,   % first generation fermions
  1/1/1/\d/\d/\d/\d/2,   % second generation fermions
  1/1/1/\d/\d/\d/\d/3,   % third generation fermions
  \d/\d/\d/1/1/1/1/0,    % bosons
  \d/\d/\d/1/1/1/\d/0,   % gauge bosons
  1/\d/\d/1/\d/\d/\d/0,  % strong interactions
  1/1/\d/\d/1/\d/\d/0,   % electromagnetic interactions
  1/1/1/\d/\d/1/\d/0,    % weak interactions
  1/1/1/\d/\d/1/1/0      % Higgs interactions
}{ % loop over opacities
\begin{tikzpicture}[fillbox=\f]
  \message{^^JSM particles: fillbox=\f, Qua=\opQua, Lep=\opLep, Nu=\opNu, 
           Glu=\opGlu, Gam=\opGam, Weak=\opWeak, Hig=\opHig}
  
  % SWITCHES
  \pgfmathsetmacro\opAllLep{max(\opLep,\opNu)}
  \pgfmathsetmacro\opGau{max(\opGlu,\opGam,\opWeak)}
  \pgfmathsetmacro\opBos{max(\opGlu,\opGam,\opWeak,\opHig)}
  
  % QUARKS
  %\begin{scope}[opacity=\opQua,pic opacity=\opQua] % to highlight others
    \pic[pic opacity=\opGen{\opQua}{1}] (QU) at (1,4) {
      particle={quarkcol}{u}{up}{%
        \simeq2.2\MeV}{\!\sfrac{+\!2}{3}}{\sfrac{1}{2}}
    };
    \pic[pic opacity=\opGen{\opQua}{2}] (QC) at (2,4) {
      particle={quarkcol}{c}{charm}{%
        \simeq1.3\GeV}{\!\sfrac{+\!2}{3}}{\sfrac{1}{2}}
    };
    \pic[pic opacity=\opGen{\opQua}{3}] (QT) at (3,4) {
      particle={quarkcol}{t}{top}{%
        \simeq173\GeV}{\!\sfrac{+\!2}{3}}{\sfrac{1}{2}}
    };
    \pic[pic opacity=\opGen{\opQua}{1}] (QD) at (1,3) {
      particle={quarkcol}{d}{down}{%
        \simeq4.7\MeV}{\!\sfrac{-\!1}{3}}{\sfrac{1}{2}}
    };
    \pic[pic opacity=\opGen{\opQua}{2}] (QS) at (2,3) {
      particle={quarkcol}{s}{strange}{%
        \simeq96\MeV}{\!\sfrac{-\!1}{3}}{\sfrac{1}{2}}
    };
    \pic[pic opacity=\opGen{\opQua}{3}] (QB) at (3,3) {
      particle={quarkcol}{b}{bottom}{%
        \simeq4.2\GeV}{\!\sfrac{-\!1}{3}}{\sfrac{1}{2}}
    };
    \node[below left,proplabel]
      at (0.5,4+0.5*\pw) {\strut mass};
    \node[below left,proplabel]
      at (0.5,4+0.35*\pw) {\strut charge};
    \node[below left,proplabel]
      at (0.5,4+0.20*\pw) {\strut spin};
    \node[quarkcol,bflabel,above right=0pt and -2pt,opacity=\opQua]
      at (QD-sw) {QUARKS};
  %\end{scope}
  
  % LEPTONS
  \pic[pic opacity=\opGen{\opLep}{1}] (EL) at (1,2) {
    particle={leptoncol}{e}{electron}{%
      \simeq0.511\MeV}{-1}{\sfrac{1}{2}}
  };
  \pic[pic opacity=\opGen{\opLep}{2}] (MU) at (2,2) {
    particle={leptoncol}{$\unslant\mu$}{muon}{%
      \simeq106\MeV}{-1}{\sfrac{1}{2}}
  };
  \pic[pic opacity=\opGen{\opLep}{3}] (TAU) at (3,2) {
    particle={leptoncol}{$\unslant\tau$}{tau}{%
      \simeq1.777\GeV}{-1}{\sfrac{1}{2}}
  };
  \pic[pic opacity=\opGen{\opNu}{1},textscale=0.66] (NE) at (1,1) {
    particle={leptoncol}{$\PGn{\text{e}}$}{electron\\[-3pt]neutrino}{%
      <1.0\eV}{0}{\sfrac{1}{2}}
  };
  \pic[pic opacity=\opGen{\opNu}{2},textscale=0.66] (NM) at (2,1) {
    particle={leptoncol}{$\PGn{\unslant\mu}$}{muon\\[-3pt]neutrino}{%
      <0.17\eV}{0}{\sfrac{1}{2}}
  };
  \pic[pic opacity=\opGen{\opNu}{3},textscale=0.66] (NT) at (3,1) {
    particle={leptoncol}{$\PGn{\unslant\tau}$}{tau\\[-3pt]neutrino}{%
      <18.2\MeV}{0}{\sfrac{1}{2}}
  };
  \node[leptoncol,bflabel,above right=0pt and -2pt,opacity=\opAllLep]
    at (NE-sw) {LEPTONS};
  
  % GAUGE BOSONS
  \begin{scope}[pic opacity=1] % to highlight others
    \pic[pic opacity=\opGlu] (GLU) at (4,4) {
      particle={gaugecol}{g}{gluon}{%
        0}{0}{1}
    };
    \pic[pic opacity=\opGam] (GAM) at (4,3) {
      particle={gaugecol}{$\gamma$}{photon}{%
        0}{0}{1}
    };
    \pic[pic opacity=\opWeak] (W) at (4,2) {
      particle={gaugecol}{$\mathrm{W}$}{W boson}{% %^\pm
        \simeq80.4\GeV}{\pm1}{1}
    };
    \pic[pic opacity=\opWeak] (Z) at (4,1) {
      particle={gaugecol}{$\mathrm{Z}$}{Z boson}{% %^0$
        \simeq91.2\GeV}{0}{1}
    };
    %%%\pic[pic opacity=\opWeak,textscale=0.7] (L) at (5.6,1) {
    %%%  particle={gaugecol}{LQ}{leptoquark}{% %^0$
    %%%    ?}{?}{1} %>1\TeV
    %%%};
  \end{scope}
  \begin{scope}[opacity=\opGau] % to highlight others
    \node[gaugecol,bflabel,below right=0pt and 2pt]
      (GB) at (Z-se) {GAUGE BOSONS};
    \node[gaugecol,bflabel,below right=-1pt and 2pt,scale=0.7]
      at (GB.south west) {VECTOR BOSONS};
  \end{scope}
  
  % SCALAR BOSONS
  \begin{scope}[opacity=\opHig,pic opacity=\opHig] % to highlight others
    \pic (HIG) at (5,4) {
      particle={scalarcol}{H}{Higgs}{%
        \simeq125\GeV}{0}{1}
    };
    \node[scalarcol,bflabel,above left=-2pt and 2pt]
      at (HIG-se) {SCALAR BOSONS};
  \end{scope}
  
  %%%% TENSOR BOSONS
  %%%\begin{scope}[opacity=\opHig,pic opacity=\opHig] % to highlight others
  %%%  \pic (GRA) at (6,4) {
  %%%    particle={tensorcol}{G}{Graviton}{%
  %%%      0}{0}{2}
  %%%  };
  %%%  \node[tensorcol,bflabel,above left=-2pt and 2pt]
  %%%    (TB) at (GRA-se) {TENSOR BOSONS};
  %%%  \node[tensorcol,bflabel,above left=-1pt and 2pt,scale=0.7]
  %%%    at (TB.north east) {HYPOTHETICAL};
  %%%\end{scope}
  
  % HEADER
  \fill[mylightblue,rounded corners=4pt]
    (1-\pw/2,4.74) rectangle (3+\pw/2,5.1)
    node[midway,header] {%
      three generations of matter\\[0pt]
      (fermions)};
  \fill[mylightblue,rounded corners=4pt]
    (4-\pw/2,4.74) rectangle (5+\pw/2,5.1)
    node[midway,header] {%
      interactions / forces\\[0pt]
      (bosons)};
  \node[above=0pt,scale=0.75] at (1,4.5) {I};
  \node[above=0pt,scale=0.75] at (2,4.5) {II};
  \node[above=0pt,scale=0.75] at (3,4.5) {III};
  
  % INTERACTION GROUPS
  \ifnum\pgfkeysvalueof{/tikz/fillbox}=0
  \begin{pgfonlayer}{back} % draw on back
    
    % STRONG INTERACTIONS
    \def\R{11.5pt}
    \fill[intgroup,opacity=0.5*\opGlu] %=blue!20!white]
      (QU-p)++(0,\R) -- ($(GLU-p)+(0,\R)$) arc(90:-90:\R)
      to[out=-180,in=90,looseness=1.2] ($(QB-p)+(\R,0)$) arc(0:-90:\R)
      -- ($(QD-p)+(0,-\R)$) arc(-90:-180:\R)
      -- ($(QU-p)+(-\R,0)$) arc(180:90:\R)
      -- cycle;
    
    % ELECTROMAGNETIC INTERACTIONS
    \def\R{13.5pt}
    \fill[intgroup,opacity=0.5*\opGam] %=green!20!white]
      (QU-p)++(0,\R) -- ($(QT-p)+(0,\R)$) arc(90:0:\R)
      to[out=-90,in=180,looseness=1.2] ($(GAM-p)+(0,\R)$) arc(90:-90:\R)
      to[out=-180,in=90,looseness=1.2] ($(TAU-p)+(\R,0)$) arc(0:-90:\R)
      -- ($(EL-p)+(0,-\R)$) arc(-90:-180:\R)
      -- ($(QU-p)+(-\R,0)$) arc(180:90:\R)
      -- cycle;
    
    % WEAK INTERACTIONS
    \def\R{15.5pt}
    \fill[intgroup,opacity=0.5*\opWeak] %=mypurple!20!white]
      (QU-p)++(0,\R) -- ($(QT-p)+(0,\R)$) arc(90:0:\R)
      -- ($(QB-p)+(\R,0)$)
      to[out=-90,in=180,looseness=1.4] ($(W-p)+(0,\R)$) arc(90:0:\R)
      -- ($(Z-p)+(\R,0)$) arc(0:-90:\R)
      -- ($(NE-p)+(0,-\R)$) arc(-90:-180:\R)
      -- ($(QU-p)+(-\R,0)$) arc(180:90:\R)
      -- cycle;
    
  \end{pgfonlayer}
  \fi
  
\end{tikzpicture}
} % close foreach loop over opacities
} % close foreach loop over \f


\end{document}