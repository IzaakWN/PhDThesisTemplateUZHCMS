% Author: Izaak Neutelings (July 2023)
% Description:
%   Meson & baryon octets/nonets and decuplets with I3-S-Q axes
% Sources:
%   https://en.wikipedia.org/wiki/Gell-Mann–Nishijima_formula
%   https://en.wikipedia.org/wiki/Quark_model
%   https://en.wikipedia.org/wiki/Eightfold_way_(physics)
%   https://commons.wikimedia.org/wiki/File:Meson-octet-small.svg
%   https://commons.wikimedia.org/wiki/File:Baryon-octet-small.svg
%   https://commons.wikimedia.org/wiki/File:Baryon-decuplet-small.svg
\documentclass[border=3pt,tikz]{standalone}
\usepackage{amsmath} % for \text
\usetikzlibrary{arrows.meta} % to control arrow size
\tikzset{>={Latex[length=4,width=4]}} % for LaTeX arrow head
\usetikzlibrary{calc}
\usepackage{xcolor} % for colored text

% COLORS
\colorlet{myblue}{blue!90!black}
\colorlet{mydarkblue}{blue!40!black}
\colorlet{myred}{red!90!black!80}
\colorlet{mygreen}{green!60!black}
\colorlet{mylightgreen}{green!75!black}

% STYLES
\renewcommand\bar[1]{\overline{\mathrm{#1}}}
\newcommand\PM[2]{$\mathrm{#1\overline{#2}}$} % meson
\tikzstyle{guide}=[ultra thin,black!15,line cap=round]
\tikzstyle{myball}=[line width=0.05,ball color=#1]
\tikzstyle{tick}=[scale=0.84,inner sep=1pt]
\tikzstyle{pname}=[mygreen,scale=1.05,inner sep=0pt,outer sep=#1pt] % particle name
\tikzstyle{lineheight}=[execute at begin node=\setlength{\baselineskip}{#1}]
\tikzstyle{particle}=[
  circle,scale=0.9,minimum size=#1,inner sep=0.5pt,outer sep=1pt, %font=\bf,
  myred,ball color=mygreen,align=center,lineheight=6pt,
  postaction={fill=green!77!black!37,fill opacity=0.8,
  draw=mygreen!50!black,line width=0.05}
]
\tikzset{
  pname/.default=0
}

% UNSLANT GREEK LETTERS for particle symbols
% https://tex.stackexchange.com/questions/145926/upright-greek-font-fitting-to-computer-modern
% https://tex.stackexchange.com/questions/236915/adjust-custom-made-upright-greek-letters-when-used-in-subscripts
\usepackage{scalerel}
\newsavebox{\foobox}
\newcommand{\slantbox}[2][0]{\mbox{%
  \sbox{\foobox}{#2}%
  \hskip\wd\foobox
  \pdfsave
  \pdfsetmatrix{1 0 #1 1}%
  \llap{\usebox{\foobox}}%
  \pdfrestore
}}
\newcommand\unslant[2][-.25]{%
  %\mkern1.2mu%
  \ThisStyle{\slantbox[#1]{$\SavedStyle#2$}}%
  \mkern-2.2mu%
}
\newcommand{\PGe}{\unslant\eta}
\newcommand{\PGp}{\unslant\pi}
\newcommand{\PGr}{\unslant\rho}
\newcommand{\PGf}{\unslant\phi}
\newcommand{\PGo}{\unslant\omega}


% SETTINGS: (default) lengths, angles, scale factors, ...
\def\r{0.9pt} % point radius
\def\Lgrid{1.45} % maximum length of grid lines
\def\ltick{1.6pt} % tick length
%\pgfmathsetmacro\xs{2/sqrt(3)}  % unit I3 isospin to make octagon with equal sides (2/tan(60))
\def\lmask{1.3*\ltick} % extend tick length to mask line
\def\lqmask{1.3*\lqtick} % extend tick length to mask line
\def\NtickS{2} % number of ticks along S axis
\def\Smin{-1}  % S axis minimum
\def\Smax{ 1}  % S axis maximum
\def\Qmin{-1}  % Q axis minimum
\def\Qmax{ 1}  % Q axis maximum
\def\Imin{-1}  % I3 axis minimum
\def\Imax{ 1}  % I3 axis maximum
\def\Qoffset{-1.75} % offset of X axis
\def\computeParameters{ % compute common parameters (angles, scales, ...)
  \pgfmathsetmacro\ang{atan(0.5)}      % angle between I3 and Q axes
  \pgfmathsetmacro\angs{atan(\xs/2)}   % angle between I3 and Q axes (x scaled) for lines
  \pgfmathsetmacro\angt{atan(0.5/\xs)} % angle Q axis title
  \pgfmathsetmacro\angl{.4*\angs+.6*\angt} % angle Q tick label
  \pgfmathsetmacro\qs{cos(\ang)}       % unit Q charge in (I3,S)
  \pgfmathsetmacro\qss{\xs*cos(\angs)} % unit Q charge in (I3,S)
  \pgfmathsetmacro\ts{\xs/sqrt(\xs^2*cos(\ang)^2+sin(\ang)^2)} % scale tick on Q axis
  \def\lqtick{\ts*\ltick} % tick length on Q axis (x scaled)
  \message{^^J- xscale xs = \xs}
  \message{^^J- angle ang = \ang}
  \message{^^J- angle angs = \angs (x-scaled)}
  \message{^^J- angle angt = \angt (for title)}
  \message{^^J- charge unit qs = \qs}
  \message{^^J- charge unit qss = \qss (x-scaled)}
}

% COMMON MACROS
\def\tick#1#2{\draw[thick] (#1) ++ (#2:\ltick) --++ (#2-180:2*\ltick)}
\def\qtick#1{ % ticks on the Q axis
  \draw[thick] % parallel to grid lines
   (\ang:\qs*#1) ++ (90+\angs:\ltick) --++ (\angs-90:2*\ltick)
  %\draw[thick] % perpendicular to Q axis
  %  (\ang:\qs*#1) ++ (90+\ang:{\lqtick/\xs} and {\lqtick})
  %  --++ (\ang-90:{2*\lqtick/\xs} and {2*\lqtick})
}
\def\particle(#1){ % small ball
  \fill[mygreen,ball color=mygreen,postaction={
    fill=mylightgreen,fill opacity=0.6,draw=mygreen!50!black,line width=0.04}]
    (#1) circle(\r)
}
\def\mesonNode(#1){\node[particle=16pt] (P) at (#1)}
\def\baryonNode(#1){\node[particle=20pt] (P) at (#1)}

% MACROS for GRIDS
\def\drawGrid#1{ % GRID for mesons
  \begin{scope}[shift={(0,#1)}] % vertical shift
    \clip[scale=1.15,rounded corners=4pt] % octagon
      (1,0) -- (1/2,1) -- (-1/2,1) -- (-1,0)
      -- (-1/2,-1) -- (1/2,-1) -- cycle;
    \foreach \i in {1,2}{
      \draw[guide] (-\i/2,-\Lgrid) -- (-\i/2,\Lgrid); % isospin I3
      \draw[guide] ( \i/2,-\Lgrid) -- ( \i/2,\Lgrid); % isospin I3
    }
    \foreach \i in {-1,0,1}{
      \draw[guide] (-\Lgrid,\i) -- (\Lgrid,\i); % strangeness S, hypercharge Y
    }
    \foreach \i [evaluate={\q=\qss*\i}] in {-1,...,1}{ % EM charge Q
      \draw[guide,rotate=\angs-90] (-\Lgrid,\q) -- (\Lgrid,\q);
    }
  \end{scope}
}
\def\drawGridBaryonTriangle{ % GRID for baryon decuplet
  \clip (-1.67,-3.2) rectangle (1.75+1.145/\xs,0.42); % prevent white space
  \begin{scope}
    \clip[shift={(0,-1)},scale=1.1,rounded corners=4pt] % octagon
      (-3/2,1) -- (1.65,1) -- (0,-2) -- cycle;
    \foreach \i in {1,2}{
      \draw[guide] (-\i/2,-\Lgrid-1) -- (-\i/2,\Lgrid-1); % isospin I3
      \draw[guide] ( \i/2,-\Lgrid-1) -- ( \i/2,\Lgrid-1); % isospin I3
    }
    \foreach \i in {-2,...,1}{
      \draw[guide] (-\Lgrid,\i-1) -- (\Lgrid,\i-1); % strangeness S
    }
    \foreach \i [evaluate={\L=1.2*(\i+6)}] in {-1,...,2}{ % EM charge Q
      \draw[guide]
        (1/2-\i,0) ++ (\ang+90:0.2*\Lgrid) --++ (\ang-90:\L);
    }
  \end{scope}
}
\def\drawFullGrid{ % GRID for offset axes
  \pgfmathsetmacro\IminGrid{\Imin+1-\Lgrid} % minimum
  \pgfmathsetmacro\ImaxGrid{\Imax-1+\Lgrid} % maximum
  \pgfmathsetmacro\SminGrid{\Smin+1-\Lgrid} % minimum
  \pgfmathsetmacro\SmaxGrid{\Smax-1+\Lgrid} % maximum
  \foreach \i [%
    parse=true,evaluate={%
      \x=\Imin+\i/2;%
      \S=max(\Smin-0.2,-\Lgrid+\Smax+2*abs(\x)-1.8);%
  }] in {0,...,2*(\Imax-\Imin)}{
    \draw[guide] % isospin I3
      (\x,\S) -- (\x,\SmaxGrid);
  }
  \foreach \i [%
    parse=true,evaluate={%
      \y=\Smin+\i;%
      \I=\ImaxGrid-0.5*abs(\y-(\Smin<-2?0:(\Smax+\Smin)/2))-0.23;%
  }] in {0,...,\Smax-\Smin}{
    \draw[guide] % strangeness S, hypercharge Y
      (\IminGrid,\y) -- (\I,\y);
  }
  \begin{scope}[shift={(0,\Smax-1)}] % vertical shift
    \clip %\draw[red]
      (\IminGrid,\Lgrid) -- (\IminGrid,\Smin-\Smax+2-\Lgrid)
      -- ($(\ang+90:\qs*\Qoffset)+(\ang:-1.1*\qs)$)
      --++ (\ang:{(\Qmax-\Qmin+0.2)*\qs}) -- (\ImaxGrid,\Lgrid) -- cycle;
    \foreach \i [parse=true,evaluate={\q=\qs*(\Qmin+\i)}] in {0,...,\Qmax-\Qmin}{
      \draw[guide] % EM charge Q
        ($(\ang+90:\qs*\Qoffset)+(\ang:\q)$) --++ (90+\ang:{(\Qoffset-\Lgrid)*(0.12*\i-1.2)});
    }
  \end{scope}
}

% MACROS for AXES
\def\drawAxisMesonOctet{ % AXES for mesons
  \draw[->,thick] (-1-0.15/\xs,0) -- (1+0.22/\xs,0) node[right=-2pt] {$I_3$};
  \draw[->,thick] (0,-1.15) -- (0,1.22) node[above=-2] {$S$};
  \draw[->,thick] (\ang:-\qs*1.15) -- (\ang:\qs*1.26)
    node[rotate=\angt,anchor=\angs+174,inner sep=0.5pt] {$Q=I_3+\frac{1}{2}S$};
  \begin{scope}[every node/.style={tick}] % ticks
    \tick{-1,0}{90} node[left=3pt,below=1pt] {$-1$};
    \tick{-1/2,0}{90}; %node[left=5pt,below] {$-\frac{1}{2}$};
    \tick{1/2,0}{90}; %node[left=3pt,below] {$+\frac{1}{2}$};
    \tick{1,0}{90} node[right=3pt,below] {$+1$};
    \tick{0,-1}{0} node[left=5pt,below=1pt] {$-1$};
    \tick{0,-1/2}{0};
    \tick{0,1/2}{0};
    \tick{0,1}{0} node[left=5pt,above=0.8pt] {$+1$};
    \qtick{1/2};
    %\qtick{-1} node[anchor=\ang+5] {$-1$}; % vertical
    \qtick{-1} node[rotate=\angl,below left=0pt] {$-1$};
    \qtick{-1/2};
    %\qtick{0} node[below right=0] {$0$}; % vertical
    \qtick{0} node[rotate=\angl,right=0.5pt,below=1pt] {$0$};
    \qtick{1/2};
    %\qtick{1} node[anchor=\ang+160] {$+1$}; % vertical
    \qtick{1} node[rotate=\angl,below left=0pt] {$+1$};
  \end{scope}
}
\def\drawAxisBaryonOctet{ % AXES for baryons octet
  \draw[->,thick] (-1-0.15/\xs,0) -- (1+0.22/\xs,0)
    node[anchor=-140,inner sep=1pt] {$I_3$};
  \draw[->,thick] (0,-2.15) -- (0,0.22)
    node[above=-2] {$S$};
  \draw[->,thick,shift={(0,-1)}] (\ang:-\qs*1.15) -- (\ang:\qs*1.26)
    node[rotate=\angt,anchor=-168,inner sep=0pt] {$Q=I_3+\frac{1}{2}(B+S)$};
  \begin{scope}[every node/.style={tick}] % ticks
    \node[below left=1pt] at (0,0) {$0$};
    \tick{-1,0}{90} node[left=3pt,below=1pt] {$-1$};
    \tick{-1/2,0}{90}; %node[left=5pt,below] {$-\frac{1}{2}$};
    \tick{1/2,0}{90}; %node[left=3pt,below] {$+\frac{1}{2}$};
    \tick{1,0}{90} node[left=3pt,below=1pt] {$+1$};
    \tick{0,-1/2}{0};
    \tick{0,-1}{0} node[above=2pt,left=0pt] {$-1$};
    \tick{0,-3/2}{0};
    \tick{0,-2}{0} node[left=5pt,below=1.0pt] {$-2$};
    \begin{scope}[shift={(0,-1)}]
      \qtick{1/2};
      %\qtick{-1} node[anchor=\ang+5] {$-1$};
      \qtick{-1} node[rotate=\angl,below left=0pt] {$-1$}; % vertical
      \qtick{-1/2};
      %\qtick{0} node[below right=0] {$0$}; % vertical
      \qtick{0} node[rotate=\angl,right=0.5pt,below=1pt] {$0$};
      \qtick{1/2};
      %\qtick{1} node[anchor=\ang+160] {$+1$}; % vertical
      \qtick{1} node[rotate=\angl,below left=0pt] {$+1$};
    \end{scope}
  \end{scope}
}
\def\drawAxisBaryonDecuplet{ % AXES for baryon decuplet
  \draw[->,thick] (-1.5-0.15/\xs,0) -- (1.5+0.25/\xs,0)
    node[anchor=-150,inner sep=2pt] {$I_3$};
  \draw[->,thick] (0,-3.15) -- (0,0.22)
    node[above=-2] {$S$};
  \draw[->,thick,shift={(0,-1)}] (\ang:-\qs*1.2) -- (\ang:\qs*2.2)
    %node[rotate=\angt,anchor=179,inner sep=2pt] {$Q=I_3+\frac{1}{2}(B+S)$};
    node[rotate=\angt,anchor=125,inner sep=12pt] {$Q=I_3+\frac{1}{2}(B+S)$};
  \begin{scope}[every node/.style={tick}] % ticks
    \node[below left=1pt] at (0,0) {$0$};
    \tick{-3/2,0}{90};
    \tick{-1,0}{90} node[left=3pt,below=1pt] at (-1,-\ltick) {$-1$};
    \tick{-1/2,0}{90}; %node[left=3pt,below] {$-\frac{1}{2}$};
    \tick{1/2,0}{90}; %node[left=3pt,below] {$+\frac{1}{2}$};
    \tick{1,0}{90} node[left=3pt,below=1pt] {$+1$};
    \tick{3/2,0}{90};
    \tick{0,-1/2}{0};
    \tick{0,-1}{0} node[above=2pt,left=0pt] {$-1$};
    \tick{0,-3/2}{0};
    \tick{0,-2}{0} node[left=1pt] {$-2$};
    \tick{0,-5/2}{0};
    \tick{0,-3}{0} node[left=1pt] {$-3$};
    \begin{scope}[shift={(0,-1)}]
      \qtick{1/2};
      %\qtick{-1} node[anchor=\ang+20,inner sep=2pt] {$-1$}; {$-1$}; % vertical
      \qtick{-1} node[rotate=\angl,below left=0pt] {$-1$};
      \qtick{-1/2};
      %\qtick{0} node[below right=0] {$0$}; % vertical
      \qtick{0} node[rotate=\angl,right=0.5pt,below=1pt] {$0$};
      \qtick{1/2};
      %\qtick{1} node[anchor=\ang+70,inner sep=2pt] {$+1$}; % vertical
      \qtick{1} node[rotate=\angl,left=3pt,below=1pt] {$+1$};
      \qtick{3/2};
      %\qtick{2} node[anchor=\ang+70,inner sep=2pt] {$+2$}; % vertical
      \qtick{2} node[rotate=\angl,left=3pt,below=1pt] {$+2$};
    \end{scope}
  \end{scope}
}
\def\drawSAxis{ % offset S AXES (vertical)
  \begin{scope}[shift={(\Imin+1-\Lgrid,0)},every node/.style={tick}] % ticks
    \draw[->,thick] (0,\Smin-0.15) -- (0,\Smax+0.22) node[above=2pt] {$S$};
    \foreach \i [parse=true,evaluate={%
      \y=\Smin+\i/2;       % tick position
      \num=int(abs(2*\y)); % numerator for denominator = 2
    }] in {0,...,2*(\Smax-\Smin)}{
      \ifodd\num % half-integer tick
        %\ifdim \y pt < 0 pt - \else + \fi % sign
        %\dfrac{\pgfmathprintnumber{\num}}{2} % fraction
      \else % whole integer tick
        \tick{0,\y}{0} node[left=1pt] {$
          \ifnum \num=0 0 \else \pgfmathprintnumber[print sign=true]{\y} \fi
        $};
      \fi
    }
  \end{scope}
}
\def\drawIAxis{ % offset I3 AXES (horizontal)
  \begin{scope}[shift={(0,\Lgrid+\Smax-1)},every node/.style={tick}] % vertical shift
    \draw[->,thick] (\Imin-0.15/\xs,0) -- (\Imax+0.22/\xs,0) node[right=2pt] {$I_3$};
    \foreach \i [parse=true,evaluate={%
      \x=\Imin+\i/2;  % tick position
      \num=int(2*\x); % numerator for denominator = 2
    }] in {0,...,2*(\Imax-\Imin)}{
      \ifodd\num % half-integer tick
        \tick{\x,0}{-90} node[left=3pt,above=0pt] {$
          \pgfmathprintnumber[print sign=true]{\num}/2 % fraction
        $};
      \else % whole integer tick
        \tick{\x,0}{-90} node[left={(\x!=0)*3pt},above=1pt] {$
          \ifnum \num=0 0 \else \pgfmathprintnumber[print sign=true]{\x} \fi
        $};
      \fi
    }
  \end{scope}
}
\def\drawQAxis{ % offset I3 AXES (diagonal)
  \begin{scope}[shift={($(\ang+90:\qs*\Qoffset)+(0,\Smax-1)$)},every node/.style={tick}] % shift
    \draw[->,thick] (\ang:{\qs*(\Qmin-0.15)}) -- (\ang:{\qs*(\Qmax+0.26)});
    \node[rotate=\angt,below=15pt] at (\ang:\qs*\Qmax) {$Q=I_3+\frac{1}{2}S$};
    \foreach \i [parse=true,evaluate={%
      \q=\Qmin+\i/2;  % tick position
      \num=int(2*\q); % numerator for denominator = 2
    }] in {0,...,2*(\Qmax-\Qmin)}{
      \ifodd\num
        \qtick{\q};
      \else % whole integer tick
        \qtick{\q} node[anchor=\ang+70,inner sep=2pt] {$
          \ifnum \num=0 \;0 \else \pgfmathprintnumber[print sign=true]{\q} \fi
        $};
      \fi
    }
    
  \end{scope}
}

% MACROS for drawing REGULAR SHAPES
\def\drawOctagon{ % MESONS OCTAGON
  \coordinate (Q-1) at (\ang:-\qs); % Q = -1
  \coordinate (Q+1) at (\ang:\qs);  % Q = +1
  \draw[myblue,very thick]
    (1,0) -- (1/2,-1) -- (\lmask,-1)
    (-\lmask,-1) -- (-1/2,-1) -- ($(Q-1)+(\angs-90:\lqmask)$)
    ($(Q-1)+(\angs+90:\lqmask)$) -- (-1,0) -- (-1/2,1) -- (-\lmask,1)
    (\lmask,1) -- (1/2,1) -- ($(Q+1)+(\angs+90:\lqmask)$)
    ($(Q+1)+(\angs-90:\lqmask)$) -- (1,0);
}
\def\drawBaryonOctagon{ % BARYON OCTAGON
  \coordinate (Q-1) at ($(0,-1)+(\ang:-\qs)$); % Q = -1
  \coordinate (Q+1) at ($(0,-1)+(\ang:\qs)$);  % Q = +1
  \draw[myblue,very thick]
    (-1/2,0) -- (1/2,0) -- ($(Q+1)+(\angs+90:\lqmask)$)
    ($(Q+1)+(\angs-90:\lqmask)$) -- (1,-1) -- (1/2,-2) -- (\lmask,-2)
    (-\lmask,-2) -- (-1/2,-2) -- ($(Q-1)+(\angs-90:\lqmask)$)
    ($(Q-1)+(\angs+90:\lqmask)$) -- (-1,-1) -- (-1/2,0);
}
\def\drawBaryonTriangle{ % BARYON TRIANGLE
  \coordinate (Q-1) at ($(0,-1)+(\ang:-\qs)$); % Q = -1
  \draw[myblue,very thick]
    (-3/2,0) -- (3/2,0) -- (0,-3) -- ($(Q-1)+(\angs-90:\lqmask)$)
    ($(Q-1)+(\angs+90:\lqmask)$) -- (-3/2,0);
}


\begin{document}


% PSEUDOSCALAR MESON OCTET / NONET
\foreach \xs [evaluate=\xs] in {2/sqrt(3)}{ % loop over xscale
\begin{tikzpicture}[scale=1.8,x={\xs cm},y=1cm]
  \message{^^JPseudoscalar meson octet (xs = \xs)}
  \computeParameters
  
  % AXES & GRID
  \drawGrid{0}
  \drawAxisMesonOctet
  
  % OCTAGON
  \drawOctagon
  
  % MESONS
  \particle(-1/2, 1) node[anchor=-70,inner sep=4pt] {K$^0$};
  \particle( 1/2, 1) node[anchor=-130,inner sep=4pt] {K$^+$};
  \particle(-1,   0) node[anchor=-70,inner sep=4pt] {$\PGp^-$};
  \particle( 1,   0) node[anchor=-130,inner sep=4pt] {$\PGp^+$};
  \particle(-1/2,-1) node[anchor=50,inner sep=2pt] {K$^-$};
  \particle( 1/2,-1) node[anchor=155,inner sep=4pt] {$\bar{K}^0$};
  \particle(\ang+30:0.09) node[anchor=-120,inner sep=4pt] {$\PGp^0$};
  %\particle(\ang/2:0.12) node[anchor=\ang/2-180,inner sep=5pt] {$\PGp^0$};
  \particle(120:0.09) node[anchor=-50,inner sep=2.5pt] {$\PGe$};
  \particle(240:0.09) node[anchor= 60,inner sep=3pt] {$\PGe'$};
  
\end{tikzpicture}} % end loop


% PSEUDOSCALAR MESON OCTET / NONET (offset axes)
\foreach \xs [evaluate=\xs] in {2/sqrt(3)}{ % loop over xscale
\begin{tikzpicture}[scale=1.7,x={\xs cm},y=1cm]
  \message{^^JPseudoscalar meson octet (xs = \xs)}
  \computeParameters
  
  % AXES & GRID
  \drawFullGrid
  \drawSAxis
  \drawIAxis
  \drawQAxis
  
  % OCTAGON
  \draw[myblue,very thick]
    (1,0) -- (1/2,-1) -- (-1/2,-1) -- (-1,0)
    -- (-1/2,1) -- (1/2,1) -- cycle;
  
  % MESONS
  \particle(-1/2, 1) node[anchor=-70,inner sep=4pt] {K$^0$};
  \particle( 1/2, 1) node[anchor=-130,inner sep=4pt] {K$^+$};
  \particle(-1,   0) node[anchor=-10,inner sep=2pt] {$\PGp^-$};
  \particle( 1,   0) node[anchor=-170,inner sep=4pt] {$\PGp^+$};
  \particle(-1/2,-1) node[anchor=50,inner sep=2pt] {K$^-$};
  \particle( 1/2,-1) node[anchor=155,inner sep=4pt] {$\bar{K}^0$};
  \particle(90:0.05) node[anchor=-100,inner sep=4pt] {$\PGp^0$};
  \particle(210:0.05) node[anchor=45,inner sep=3pt] {$\PGe$};
  \particle(330:0.05) node[anchor=150,inner sep=3pt] {$\PGe'$};
  
\end{tikzpicture}} % end loop


% PSEUDOSCALAR MESON OCTET / NONET (offset axes, quarks)
% https://commons.wikimedia.org/wiki/File:Meson-octet-small.svg
\foreach \xs [evaluate=\xs] in {2/sqrt(3)}{ % loop over xscale
\begin{tikzpicture}[scale=1.7,x={\xs cm},y=1cm]
  \message{^^JPseudoscalar meson octet (xs = \xs)}
  \computeParameters
  
  % AXES & GRID
  \drawFullGrid
  \drawSAxis
  \drawIAxis
  \drawQAxis
  
  % OCTAGON
  \draw[myblue,very thick]
    (1,0) -- (1/2,-1) -- (-1/2,-1) -- (-1,0)
    -- (-1/2,1) -- (1/2,1) -- cycle;
  
  % MESONS
  \mesonNode(-1/2, 1) {\PM{d}{u}}
    node[pname=0,anchor=-60] at (P.120) {K$^0$};
  \mesonNode( 1/2, 1) {\PM{d}{u}}
    node[pname=1.5,anchor=-140] at (P.40) {K$^+$};
  \mesonNode(-1,   0) {\PM{d}{u}}
    node[pname=-1,anchor=-60] at (P.120) {$\PGp^-$};
  \mesonNode( 1,   0) {\PM{d}{u}}
    node[pname=1,anchor=-140] at (P.40) {$\PGp^+$};
  \mesonNode(-1/2,-1) {\PM{d}{u}}
    node[pname,anchor=40] at (P.-140) {K$^-$};
  \mesonNode( 1/2,-1) {\PM{d}{u}}
    node[pname,anchor=150] at (P.-30) {$\bar{K}^0$};
  \node[particle=25pt] (P) at (0,0) {}
    node[pname=1.4,anchor=-100] at (P.70) {$\PGp^0$}
    node[pname=0.8,anchor=45] at (P.-145) {$\PGe$}
    node[pname=0.5,anchor=150] at (P.-30) {$\PGe'$};
  \node[myred,scale=0.9,anchor=-68,inner sep=1.9pt]
    at (P) {\PM{d}{d}};
  \node[myred,scale=0.9,anchor=40,inner sep=0.3pt]
    at (P) {\PM{u}{u}};
  \node[myred,scale=0.9,anchor=172,inner sep=2.7pt]
    at (P) {\PM{s}{s}};
  
\end{tikzpicture}} % end loop


% VECTOR MESON OCTET / NONET
\foreach \xs [evaluate=\xs] in {2/sqrt(3)}{ % loop over xscale
\begin{tikzpicture}[scale=1.8,x={\xs cm},y=1cm]
  \message{^^J^^JVector meson octet (xs = \xs)}
  \computeParameters
  
  % AXES & GRID
  \drawGrid{0}
  \drawAxisMesonOctet
  
  % OCTAGON
  \drawOctagon
  
  % MESONS
  \particle(-1/2, 1) node[anchor=-70,inner sep=4pt] {K$^{*0}$};
  \particle( 1/2, 1) node[anchor=-130,inner sep=4pt] {K$^{*+}$};
  \particle(-1,   0) node[anchor=-70,inner sep=4pt] {$\PGr^-$};
  \particle( 1,   0) node[anchor=-130,inner sep=4pt] {$\PGr^+$};
  \particle(-1/2,-1) node[anchor=60,inner sep=4pt] {K$^{*-}$};
  \particle( 1/2,-1) node[anchor=150,inner sep=3pt] {$\bar{K}^{*0}$};
  \particle(\ang+30:0.09) node[anchor=-130,inner sep=3pt] {$\PGr^0$};
  \particle(120:0.09) node[anchor=-50,inner sep=2pt] {$\PGf$};
  \particle(240:0.09) node[anchor=58,inner sep=3pt] {$\PGo$};
  
\end{tikzpicture}} % end loop


% VECTOR MESON OCTET / NONET (offset axes)
\foreach \xs [evaluate=\xs] in {2/sqrt(3)}{ % loop over xscale
\begin{tikzpicture}[scale=1.7,x={\xs cm},y=1cm]
  \message{^^JPseudoscalar meson octet (xs = \xs)}
  \computeParameters
  
  % AXES & GRID
  \drawFullGrid
  \drawSAxis
  \drawIAxis
  \drawQAxis
  
  % OCTAGON
  \draw[myblue,very thick]
    (1,0) -- (1/2,-1) -- (-1/2,-1) -- (-1,0)
    -- (-1/2,1) -- (1/2,1) -- cycle;
  
  % MESONS
  \particle(-1/2, 1) node[anchor=-70,inner sep=4pt] {K$^{*0}$};
  \particle( 1/2, 1) node[anchor=-130,inner sep=4pt] {K$^{*+}$};
  \particle(-1,   0) node[anchor=-10,inner sep=3pt] {$\PGr^-$};
  \particle( 1,   0) node[anchor=-170,inner sep=4pt] {$\PGr^+$};
  \particle(-1/2,-1) node[anchor=60,inner sep=4pt] {K$^{*-}$};
  \particle( 1/2,-1) node[anchor=150,inner sep=3pt] {$\bar{K}^{*0}$};
  \particle(90:0.05) node[anchor=-100,inner sep=4pt] {$\PGr^0$};
  \particle(210:0.05) node[anchor=45,inner sep=3pt] {$\PGo$};
  \particle(330:0.05) node[anchor=140,inner sep=3pt] {$\PGf$};
  
\end{tikzpicture}} % end loop


% VECTOR MESON OCTET / NONET (offset axes, quarks)
% https://commons.wikimedia.org/wiki/File:Meson-octet-small.svg
\foreach \xs [evaluate=\xs] in {2/sqrt(3)}{ % loop over xscale
\begin{tikzpicture}[scale=1.7,x={\xs cm},y=1cm]
  \message{^^JPseudoscalar meson octet (xs = \xs)}
  \computeParameters
  
  % AXES & GRID
  \def\Qoffset{-1.8}
  \drawFullGrid
  \drawSAxis
  \drawIAxis
  \drawQAxis
  
  % OCTAGON
  \draw[myblue,very thick]
    (1,0) -- (1/2,-1) -- (-1/2,-1) -- (-1,0)
    -- (-1/2,1) -- (1/2,1) -- cycle;
  
  % MESONS
  \mesonNode(-1/2, 1) {\PM{d}{u}}
    node[pname=0,anchor=-50] at (P.130) {K$^{*0}$};
  \mesonNode( 1/2, 1) {\PM{d}{u}}
    node[pname=1.5,anchor=-150] at (P.30) {K$^{*+}$};
  \mesonNode(-1,   0) {\PM{d}{u}}
    node[pname=-1,anchor=-60] at (P.120) {$\PGr^-$};
  \mesonNode( 1,   0) {\PM{d}{u}}
    node[pname=1,anchor=-140] at (P.40) {$\PGr^+$};r
  \mesonNode(-1/2,-1) {\PM{d}{u}}
    node[pname,anchor=40] at (P.-140) {K$^{*-}$};
  \mesonNode( 1/2,-1) {\PM{d}{u}}
    node[pname,anchor=154] at (P.-26) {$\bar{K}^{*0}$};
  \node[particle=25pt] (P) at (0,0) {}
    node[pname=1.2,anchor=-100] at (P.70) {$\PGr^0$}
    node[pname=0.8,anchor=45] at (P.-145) {$\PGo$}
    node[pname=0.0,anchor=145] at (P.-35) {$\PGf$};
  \node[myred,scale=0.9,anchor=-68,inner sep=1.9pt]
    at (P) {\PM{d}{d}};
  \node[myred,scale=0.9,anchor=40,inner sep=0.3pt]
    at (P) {\PM{u}{u}};
  \node[myred,scale=0.9,anchor=172,inner sep=2.7pt]
    at (P) {\PM{s}{s}};
  
\end{tikzpicture}} % end loop


% BARYON OCTET
\foreach \xs [evaluate=\xs] in {2/sqrt(3)}{ % loop over xscale
\begin{tikzpicture}[scale=1.8,x={\xs cm},y=1cm]
  \message{^^J^^JBaryon octet (xs = \xs)}
  \computeParameters
  
  % AXES & GRID
  \drawGrid{-1}
  \drawAxisBaryonOctet
  
  % OCTAGON
  \drawBaryonOctagon
  
  % BARYONS
  \particle(-1/2, 0) node[anchor=-70,inner sep=4pt] {n$^0$};
  \particle( 1/2, 0) node[anchor=-140,inner sep=3pt] {p$^+$};
  \particle(-1,  -1) node[anchor=-20,inner sep=2pt] {$\Sigma^-$};
  \particle({$(60:0.09)+(0,-1)$})
    node[anchor=-130,inner sep=3pt] {$\Sigma^0$};
  \particle({$(-120:0.09)+(0,-1)$})
    node[anchor=60,inner sep=3pt] {$\Lambda^0$};
  \particle( 1,  -1) node[anchor=-160,inner sep=4pt] {$\Sigma^+$};
  \particle(-1/2,-2) node[anchor=50,inner sep=2pt] {$\Xi^-$};
  \particle( 1/2,-2) node[anchor=140,inner sep=2pt] {$\Xi^0$};
  
\end{tikzpicture}} % end loop


% PSEUDOSCALAR BARYON OCTET (offset axes)
\foreach \xs [evaluate=\xs] in {2/sqrt(3)}{ % loop over xscale
\begin{tikzpicture}[scale=1.7,x={\xs cm},y=1cm]
  \message{^^J^^JBaryon octet (xs = \xs), offset}
  \computeParameters
  
  % AXES & GRID
  \def\Smin{-2} \def\Smax{0} % S axis limits
  \def\Imin{-1} \def\Imax{1} % I3 axis limits
  \def\Qmin{-1} \def\Qmax{1} % Q axis limits
  \drawFullGrid
  \drawSAxis
  \drawIAxis
  \drawQAxis
  
  % OCTAGON
  \draw[myblue,very thick]
    (1,-1) -- (1/2,-2) -- (-1/2,-2) -- (-1,-1)
    -- (-1/2,0) -- (1/2,0) -- cycle;
  
  % BARYONS
  \particle(-1/2, 0) node[anchor=-70,inner sep=4pt] {n$^0$};
  \particle( 1/2, 0) node[anchor=-140,inner sep=3pt] {p$^+$};
  \particle(-1,  -1) node[anchor=-10,inner sep=3pt] {$\Sigma^-$};
  \particle({$(60:0.05)+(0,-1)$})
    node[anchor=-120,inner sep=4pt] {$\Sigma^0$};
  \particle({$(-120:0.05)+(0,-1)$})
    node[anchor=80,inner sep=4pt] {$\Lambda^0$};
  \particle( 1,  -1) node[anchor=-170,inner sep=4pt] {$\Sigma^+$};
  \particle(-1/2,-2) node[anchor=50,inner sep=2pt] {$\Xi^-$};
  \particle( 1/2,-2) node[anchor=140,inner sep=2pt] {$\Xi^0$};
  
\end{tikzpicture}} % end loop


% PSEUDOSCALAR BARYON OCTET (offset axes)
\foreach \xs [evaluate=\xs] in {2/sqrt(3)}{ % loop over xscale
\begin{tikzpicture}[scale=1.7,x={\xs cm},y=1cm]
  \message{^^J^^JBaryon octet (xs = \xs), offset}
  \computeParameters
  
  % AXES & GRID
  \def\Smin{-2} \def\Smax{0} % S axis limits
  \def\Imin{-1} \def\Imax{1} % I3 axis limits
  \def\Qmin{-1} \def\Qmax{1} % Q axis limits
  \def\Qoffset{-1.8}
  \drawFullGrid
  \drawSAxis
  \drawIAxis
  \drawQAxis
  
  % OCTAGON
  \draw[myblue,very thick]
    (1,-1) -- (1/2,-2) -- (-1/2,-2) -- (-1,-1)
    -- (-1/2,0) -- (1/2,0) -- cycle;
  
  % BARYONS
  \baryonNode(-1/2, 0) {dd\\u}
    node[pname=0,anchor=-70] at (P.120) {n$^0$};
  \baryonNode( 1/2, 0) {uu\\d}
    node[pname=1,anchor=-130] at (P.40) {p$^+$};
  \baryonNode(-1,  -1) {dd\\d}
    node[pname=-1,anchor=-50] at (P.140) {$\Sigma^-$};
  \baryonNode( 0,  -1) {ud\\s}
    node[pname=2,anchor=-120] at (P.50) {$\Sigma^0$}
    node[pname=0,anchor=40] at (P.-120) {$\Lambda^0$};
  \baryonNode( 1,  -1) {uu\\s}
    node[pname=1,anchor=-140] at (P.30) {$\Sigma^+$};
  \baryonNode(-1/2,-2) {ds\\s}
    node[pname=-1,anchor=30] at (P.-150) {$\Xi^-$};
  \baryonNode( 1/2,-2) {us\\s}
    node[pname,anchor=150] at (P.-30) {$\Xi^0$};
  
\end{tikzpicture}} % end loop


% BARYON DECUPLET
\foreach \xs [evaluate=\xs] in {2/sqrt(3)}{ % loop over xscale
\begin{tikzpicture}[scale=1.8,x={\xs cm},y=1cm]
  \message{^^J^^JBaryon decuplet (xs = \xs)}
  \computeParameters
  
  % AXES & GRID
  \drawGridBaryonTriangle
  \drawAxisBaryonDecuplet
  
  % TRIANGLE
  \drawBaryonTriangle
  
  % BARYONS
  \particle(-3/2, 0) node[anchor=-110,inner sep=4pt] {$\Delta^-$};
  \particle(-1/2, 0) node[anchor=-100,inner sep=4pt] {$\Delta^0$};
  \particle( 1/2, 0) node[anchor=-115,inner sep=4pt] {$\Delta^+$};
  \particle( 3/2, 0) node[anchor=-80,inner sep=4pt] {$\Delta^{++}$};
  \particle(-1,  -1) node[anchor=25,inner sep=1pt] {$\Sigma^{*-}$};
  \particle( 0,  -1) node[anchor=-136,inner sep=5pt] {$\Sigma^{*0}$};
  \particle( 1,  -1) node[anchor=160,inner sep=4pt] {$\Sigma^{*+}$};
  \particle(-1/2,-2) node[anchor=25,inner sep=1pt] {$\Xi^{*-}$};
  \particle( 1/2,-2) node[anchor=160,inner sep=4pt] {$\Xi^{*0}$};
  \particle( 0,  -3) node[anchor=160,inner sep=4pt] {$\Omega^-$};
  
\end{tikzpicture}} % end loop


% BARYON DECUPLET (offset axes)
\foreach \xs [evaluate=\xs] in {2/sqrt(3)}{ % loop over xscale
\begin{tikzpicture}[scale=1.7,x={\xs cm},y=1cm]
  \message{^^J^^JBaryon decuplet (xs = \xs)}
  \computeParameters
  
  % AXES & GRID
  \def\Smin{-3}   \def\Smax{0} % S axis limits
  \def\Imin{-1.5} \def\Imax{1.5} % I3 axis limits
  \def\Qmin{-1}   \def\Qmax{2} % Q axis limits
  \def\Qoffset{-2.45}
  \drawFullGrid
  \drawSAxis
  \drawIAxis
  \drawQAxis
  
  % OCTAGON
  \draw[myblue,very thick]
    (-3/2,0) -- (3/2,0) -- (0,-3) -- cycle;
  
  % BARYONS
  \particle(-3/2, 0) node[anchor=-90,inner sep=4pt] {$\Delta^-$};
  \particle(-1/2, 0) node[anchor=-100,inner sep=4pt] {$\Delta^0$};
  \particle( 1/2, 0) node[anchor=-115,inner sep=4pt] {$\Delta^+$};
  \particle( 3/2, 0) node[anchor=-120,inner sep=4pt] {$\Delta^{++}$};
  \particle(-1,  -1) node[anchor=25,inner sep=1pt] {$\Sigma^{*-}$};
  \particle( 0,  -1) node[anchor=120,inner sep=3pt] {$\Sigma^0$};
  \particle( 1,  -1) node[anchor=160,inner sep=4pt] {$\Sigma^{*+}$};
  \particle(-1/2,-2) node[anchor=25,inner sep=1pt] {$\Xi^{*-}$};
  \particle( 1/2,-2) node[anchor=160,inner sep=4pt] {$\Xi^{*0}$};
  \particle( 0,  -3) node[anchor=110,inner sep=4pt] {$\Omega^-$};
  
\end{tikzpicture}} % end loop


% BARYON DECUPLET (offset axes)
\foreach \xs [evaluate=\xs] in {2/sqrt(3)}{ % loop over xscale
\begin{tikzpicture}[scale=1.7,x={\xs cm},y=1cm]
  \message{^^J^^JBaryon decuplet (xs = \xs)}
  \computeParameters
  
  % AXES & GRID
  \def\Smin{-3}   \def\Smax{0} % S axis limits
  \def\Imin{-1.5} \def\Imax{1.5} % I3 axis limits
  \def\Qmin{-1}   \def\Qmax{2} % Q axis limits
  \def\Qoffset{-2.45}
  \drawFullGrid
  \drawSAxis
  \drawIAxis
  \drawQAxis
  
  % OCTAGON
  \draw[myblue,very thick]
    (-3/2,0) -- (3/2,0) -- (0,-3) -- cycle;
  
  % BARYONS
  \baryonNode(-3/2, 0) {dd\\d}
    node[pname,anchor=-60] at (P.130) {$\Delta^-$};
  \baryonNode(-1/2, 0) {dd\\u}
    node[pname,anchor=-60] at (P.130) {$\Delta^0$};
  \baryonNode( 1/2, 0) {uu\\d}
    node[pname=1,anchor=-145] at (P.40) {$\Delta^+$};
  \baryonNode( 3/2, 0) {uu\\u}
    node[pname=2,anchor=-150] at (P.30) {$\Delta^{++}$};
  \baryonNode(-1,  -1) {dd\\s}
    node[pname,anchor=20] at (P.-160) {$\Sigma^{*-}$};
  \baryonNode( 0,  -1) {ud\\s}
    node[pname,anchor=160] at (P.-30) {$\Sigma^0$};
  \baryonNode( 1,  -1) {uu\\s}
    node[pname,anchor=160] at (P.-20) {$\Sigma^{*+}$};
  \baryonNode(-1/2,-2) {ds\\s}
    node[pname,anchor=20] at (P.-160) {$\Xi^{*-}$};
  \baryonNode( 1/2,-2) {us\\s}
    node[pname,anchor=160] at (P.-20) {$\Xi^{*0}$};
  \baryonNode( 0,  -3) {ss\\s}
    node[pname,anchor=60] at (P.-120) {$\Omega^-$};
  
\end{tikzpicture}} % end loop


\end{document}
