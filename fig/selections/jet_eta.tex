% Author: Izaak Neutelings (October 2021)
% Description: Rapidity regions of dijet events
% Inspiration:
%   Daniel Savoiu, https://indico.cern.ch/event/1073619/#5-update-on-dijet-production-s
\documentclass[border=3pt,tikz]{standalone}
\usepackage{amsmath}
\usepackage{listofitems} % for \readlist to create arrays
\usepackage{xcolor}
\usepackage[outline]{contour} % glow around text
\usetikzlibrary{calc}
\usetikzlibrary{math} % for \tikzmath
\contourlength{1.0pt}
\tikzset{>=latex} % for LaTeX arrow head

\colorlet{myred}{red!80!black}
\colorlet{mypurple}{blue!50!red}
\colorlet{myorange}{orange!90!red!95!black!90}
\colorlet{mydarkblue}{blue!60!black}
\colorlet{mygray}{blue!20!black!30}
\tikzstyle{muon}=[myred,line width=0.6,line cap=round]
\tikzstyle{boson}=[dash pattern=on 0.2 off 0.5,line width=0.3,line cap=round,myorange!90!black]
\readlist\cols{blue,green,myorange,red,mypurple} % list of colors

\newcommand\jetcone[5][blue]{{
  \pgfmathanglebetweenpoints{\pgfpointanchor{#2}{center}}{\pgfpointanchor{#3}{center}}
  \edef\ang{#4/2} % half-opening angle
  \edef\e{#5} % ratio a/b ("eccentricity") of cone top
  \edef\vang{\pgfmathresult} % angle of vector OV
  \tikzmath{
    coordinate \C;
    \C = (#2)-(#3); % vector OV
    \x = veclen(\Cx,\Cy)*\e*sin(\ang)^2; % x coordinate P
    \y = tan(\ang)*(veclen(\Cx,\Cy)-\x); % y coordinate P
    \a = veclen(\Cx,\Cy)*sqrt(\e)*sin(\ang); % vertical radius
    \b = veclen(\Cx,\Cy)*tan(\ang)*sqrt(1-\e*sin(\ang)^2); % horizontal radius
    \angb = acos(sqrt(\e)*sin(\ang)); % angle of P in ellipse
  }
  \coordinate (tmpL) at ($(#3)-(\vang:\x pt)+(\vang+90:\y pt)$); % tangency
  \draw[thin,#1!40!black,rotate=\vang, %,fill=#1!50!black!80
    top color=#1!60!black!90,bottom color=#1!70!black!75,shading angle=\vang]
    (#3) ellipse({\a pt} and {\b pt});
  \draw[thin,#1!40!black,rotate=\vang,rounded corners=0.03,
  top color=#1!90!black!40,bottom color=#1!40!black!60,shading angle=\vang]
    (tmpL) arc(180-\angb:180+\angb:{\a pt} and {\b pt})
    -- (#2) -- cycle;
}}
\def\tick#1#2{\draw[thick] (#1) ++ (#2:0.1) --++ (#2-180:0.2)}


\begin{document}


% DIJET EVENTS
% Daniel Savoiu, https://indico.cern.ch/event/1073619/#5-update-on-dijet-production-s
\begin{tikzpicture}[scale=0.8]
  
  \def\N{5} % number of eta bins with width 0.5
  
  % DIAGONAL GRIDS (ymax)
  \foreach \i [evaluate={\ym=0.5*(\i-1);}] in {1,...,\N}{
    \draw[mygray]
      (\i-1,0) --++ (135:{sqrt(2)*(\i+0.2)}) coordinate (T\i) --++ (45:{sqrt(2)/2});
    \draw[mygray] (T\i) --++ (135:0.06) node[above=-2,rotate=45] {\ym};
  }
  \draw[mygray] (\N,0) --++ (135:{sqrt(2)*(\N+0.75)}) node[right=3,above=0,rotate=45] {2.5};
  
  % GRIDS
  \foreach \i [evaluate={\y=\i-0.5; \Nx=\N-\i+1; \e=0.5*\i}] in {1,...,\N}{
    \draw[very thin,dashed,black!50] (0,\i) --++ (\Nx,0);
    \draw[very thin,dashed,black!50] (\i,0) --++ (0,\Nx);
    \tick{0,\i}{0} node[left=-1,scale=0.9] {\e};
    \tick{\i,0}{90} node[below=-1,scale=0.9] {\e};
  }
  
  % JETS
  \foreach \i [evaluate={\y=\i-0.5; \Nx=\N-\i+1; \e=0.5*\i}] in {1,...,\N}{
    \foreach \j [evaluate={\x=\j-0.5; \mix=100-60*(\j-1)/(\N-1);
                           \ang=60*(\i-1)/(\N-1); % seperation
                           \dang=90-60*(\j-1)/(\N-1); % boost
                 }] in {1,...,\Nx}{
      \colorlet{conecol}{\cols[\i]!\mix} % color for cone
      \coordinate (O) at (\x,\y); % primary vertex
      \coordinate (T) at ($(\x,\y)+(\ang+\dang:0.42)$); % top jet
      \coordinate (B) at ($(\x,\y)+(\ang-\dang:0.42)$); % bottom jet
      \jetcone[conecol]{O}{T}{30}{0.06} %!70!white
      \jetcone[conecol]{O}{B}{30}{0.16}
    }
  }
  
  % AXIS
  \draw[<->,thick]
    (0,1.1*\N) %node[left] {\contour{white}{$y*$}} --
    node[left=6,above right=-3] {$y^* = \frac{1}{2}|y_1-y_2|$} --
    (0,0) -- (1.1*\N,0) node[right] {$y_\text{b} = \frac{1}{2}|y_1+y_2|$};
  \draw[->,thick,mygray]
    (135:4.0) --++ (45:1.3)
    node[pos=0.7,above,rotate=45] {$|y_\text{max}|$};
  
\end{tikzpicture}


% Z -> mumu recoil against jet
% Thomas Berge, https://publish.etp.kit.edu/record/21710
\begin{tikzpicture}[scale=0.8]
  
  \def\N{5} % number of eta bins with width 0.5
  
  % DIAGONAL GRIDS (ymax)
  \foreach \i [evaluate={\ym=0.5*(\i-1);}] in {1,...,\N}{
    \draw[mygray]
      (\i-1,0) --++ (135:{sqrt(2)*(\i+0.2)}) coordinate (T\i) --++ (45:{sqrt(2)/2});
    \draw[mygray] (T\i) --++ (135:0.06) node[above=-2,rotate=45] {\ym};
  }
  \draw[mygray] (\N,0) --++ (135:{sqrt(2)*(\N+0.75)}) node[right=3,above=0,rotate=45] {2.5};
  
  % GRIDS
  \foreach \i [evaluate={\y=\i-0.5; \Nx=\N-\i+1; \e=0.5*\i}] in {1,...,\N}{
    \draw[very thin,dashed,black!50] (0,\i) --++ (\Nx,0);
    \draw[very thin,dashed,black!50] (\i,0) --++ (0,\Nx);
    \tick{0,\i}{0} node[left=-1,scale=0.9] {\e};
    \tick{\i,0}{90} node[below=-1,scale=0.9] {\e};
  }
  
  % JETS
  \foreach \i [evaluate={\y=\i-0.5; \Nx=\N-\i+1; \e=0.5*\i}] in {1,...,\N}{
    \foreach \j [evaluate={\x=\j-0.5; \mix=100-60*(\j-1)/(\N-1);
                           \ang=60*(\i-1)/(\N-1); % seperation
                           \dang=90-60*(\j-1)/(\N-1); % boost
                 }] in {1,...,\Nx}{
      \colorlet{conecol}{\cols[\i]!\mix} % color for cone
      \coordinate (O) at (\x,\y); % primary vertex
      \coordinate (T) at ($(\x,\y)+(\ang+\dang:0.12)$); % Z -> mumu
      \coordinate (B) at ($(\x,\y)+(\ang-\dang:0.42)$); % jet cone
      \jetcone[conecol]{O}{B}{30}{0.16}
      \draw[boson] (O) -- (T);
      \draw[muon] (T) to[out=\ang+\dang+10,in=\ang+\dang-150]++ (\ang+\dang+20:0.30);
      \draw[muon] (T) to[out=\ang+\dang-20,in=\ang+\dang+130]++ (\ang+\dang-33:0.32);
    }
  }
  
  % LEGEND
  \begin{scope}[shift={(5,4)},scale=1.2]
    \def\ang{120}
    \coordinate (O) at (0,0); % primary vertex
    \coordinate (T) at (\ang:0.12); % muons
    \coordinate (B) at (\ang+180:0.42); % jet cone
    %\colorlet{conecol}{\cols[0]} % color for cone
    \draw[myorange!60!yellow!90,very thin] (-0.5,0) -- (0.5,0)
      node[above=1,right,scale=0.9] {beam axis};
    \jetcone[blue]{O}{B}{30}{0.16}
    \draw[boson] (O) -- (T);
    \draw[muon] (T) to[out=\ang+10,in=\ang-150]++ (\ang+20:0.30);
    \draw[muon] (T) to[out=\ang-20,in=\ang+130]++ (\ang-33:0.32) coordinate (M);
    \node[anchor=185,inner sep=3,myred,scale=0.9] at (M) {$\mathrm{Z}\to\mu\mu$};
    \node[anchor=175,inner sep=5,mydarkblue,scale=0.9] at (B) {recoil jet};
  \end{scope}
  
  % AXIS
  \draw[<->,thick]
    (0,1.1*\N) %node[left] {\contour{white}{$y*$}} --
    node[left=6,above right=-3] {$y^* = \frac{1}{2}|y_1-y_2|$} --
    (0,0) -- (1.1*\N,0) node[right] {$y_\text{b} = \frac{1}{2}|y_1+y_2|$};
  \draw[->,thick,mygray]
    (135:4.0) --++ (45:1.3)
    node[pos=0.7,above,rotate=45] {$|y_\text{max}|$};
  
\end{tikzpicture}


\end{document}